\documentclass{article}%
\usepackage[T1]{fontenc}%
\usepackage[utf8]{inputenc}%
\usepackage{lmodern}%
\usepackage{textcomp}%
\usepackage{lastpage}%
\usepackage{geometry}%
\geometry{margin=2.5cm,includeheadfoot=true}%
\usepackage[german]{babel}%
\usepackage{array}%
\usepackage{longtable}%
\usepackage{booktabs}%
\usepackage{multirow}%
\usepackage{hyperref}%
\usepackage{tabularx}%
\usepackage{xcolor}%
\usepackage{graphicx}%
\usepackage{url}%
%
\setlength{\parindent}{0pt}%
%
\begin{document}%
\normalsize%
\begin{longtable}{p{0.3\linewidth}}
\toprule
Anmeldung zur Prüfung des studienabschließenden Moduls (UdK) im SoSe 2025 \\
\midrule
Terminplan: \\
Anmeldung zur Prüfung des studienabschließenden Moduls im IPA & Digitale Einreichung der Anträge beim IPA, Frau Matrousian stud31@intra.udk-berlin.de & bis Fr, 21.02.2025 \\
Einreichen des Themas der Masterprojekt & Einverständnis der einzelnen Prüfer*innen muss durch den*die Antragsteller*in eingeholt werden (siehe Antrag). \newline  Das Thema des Masterprojekts muss mit der*dem Modulverantwortlichen abgesprochen sein. Nur vollständig ausgefüllte Anmeldeformulare (Formular Center UdK) mit allen Anlagen gem. §17 Prüfungsordnung werden angenommen. \newline  Bei Nichteinhaltung dieses Termins erfolgt ohne Ausnahme keine Zulassung zur Prüfung. \\
Zulassungsentscheidung Prüfungsausschuss & Fr, 07.03.2025 \\
Themenausgabe durch das IPA & Fr, 14.03.2025 \\
Letztmögliche Abgabe fehlender Modulblätter & Abgabe im IPA, persönlich oder per Post, Eingangsstempel der UdK ist maßgebend & bis Fr, 25.07.2025 \\
Abgabe des schriftlichen Anteils der Arbeit & 3 Exemplare bei der Studiengangsleitung \newline  Design \& Computation c/o Präsident der UdK Einsteinufer 43 10587 Berlin \newline  bis 12:00 Uhr und eine digitale Version an al@design-computation.berlin \newline  Bei Nichteinhaltung dieses Termins erfolgt ohne Ausnahme keine Prüfung. & bis Do, 02.10.2025 \\
Präsentation und mündliche Prüfung & Die Prüfungstermine werden vom IPA ca. eine Woche vor dem Prüfungszeitraum verschickt. & 20.10.2025 bis 31.10.2025 \\
\bottomrule
\end{longtable}
%
\section*{Merkblatt Design \& Computation - Masterprüfung (UdK), Anmeldung zum SoSe 2025 }
%
Bitte beachten Sie die Studien- und Prüfungsordnung in ihrer jeweils gültigen Fassung.
%
\subsection*{Anmeldung und Antragsformular:}
%
\begin{itemize}
\item Die Fristen finden Sie im Terminplan (siehe Formular-Center der UdK):
\end{itemize}
%
Termine sind Ausschlussfristen!
%
\begin{itemize}
\item Die Einreichung aller Unterlagen (Antrag inkl. entsprechender Modulblätter) erfolgt als eine PDF per E-Mail an das IPA der UdK über stud31@intra.udk-berlin.de
\item Der Prüfungsausschuss prüft und entscheidet im Anschluss über die eingegangenen Anträge.
\end{itemize}
%
Sie erhalten entsprechend eine Rückmeldung über das IPA.
%
\subsection*{Wahl der Prüfer*innen / Prüfungskommission:}
%
\begin{itemize}
\item Die Prüfungskommission besteht aus min. zwei Prüfer*innen, es besteht auch die Möglichkeit drei Prüfer*innen zu wählen.
\end{itemize}
\$\\rightarrow\$ Gewählt werden können Hochschullehrer*innen, hauptberufliche tätige Lehrkräfte mit selbstständiger Lehre oder Lehrbeauftragte der UdK oder TU. In Ausnahmefällen können auch Lehrpersonen aus anderen Hochschulen, soweit der PA zustimmt. Hochschullehrende sind dabei zu bevorzugen. Prüfungsberechtigt sind alle Hochschullehrenden im Rahmen des Fachgebiets Design \\\& Computation.
Dem PA steht es frei zusätzlich an der Prüfung teilzunehmen und diese zu beobachten (Gewährleistung der ordentlichen Durchführung der Prüfung).
\$\\rightarrow\$ Der*Die Kandidat*in hat hierbei ein Vorschlagsrecht. Der Prüfungsausschuss soll vom Vorschlag des* \$r\$ Kandidaten*in nur in begründeten Fällen, insbesondere zur Sicherstellung einer gleichmäßigen Prüfungsbelastung, abweichen.
\begin{itemize}
\item Die gewählten Prüfer*innen müssen ihr Einverständnis zur Betreuung durch Unterschriften auf dem Antragsformular oder mittels einer Bestätigungsmail erklären. E-Mail-Bestätigungen sind dem Antrag beizufügen.
\end{itemize}
%

\subsection*{Modulblätter (Scheine):}
%
\begin{itemize}
\item Die entsprechenden Modulblätter müssen vollständig ausgefüllt, von den Lehrkräften und abschließend von den Modulverantwortlichen unterschrieben und ggf. benotet sein.
\item Bei fehlenden Modulblättern können zu Prüfende nur unter Vorbehalt (u.V.) zur Prüfung zugelassen werden, Nachreichfristen siehe Terminplan.
\item Anerkennungen von einzelnen Studienleistungen oder ganzen Modulen, die durch einen Hochschulwechsel oder im Auslandssemestern erbracht worden sind, sind vorab beim Prüfungsausschuss auf den entsprechenden Modulblättern einzuholen und mit einer Kopie des Nachweises (z.B. Transcript of Records) beizulegen.
\end{itemize}
\$\\rightarrow\$ Falls im Moment keine Originalunterschriften der Lehrenden möglich sind, bündelt der*die Student*in alle Bestätigungen (Lehrveranstaltungen + Modulprüfung, auch als Mailbestätigung möglich) je Modulblatt in einem PDF und schickt es ans IPA. Als Deckblatt für die Bestätigungen kann das jeweilige Modulblatt benutzt werden, mit den eingetragenen Namen der Lehrveranstaltungen und den Namen der Lehrenden; dahinter folgen die einzelnen Bestätigungen.
%

\subsection*{Änderungen von Titeln der gestalterischen-künstlerischen-wissenschaftlichen Arbeit und von Prüfern*innen:}
%
\begin{itemize}
\item formloser Antrag per E-Mail mit dem jeweiligen Änderungswunsch zzgl. Bestätigung der Prüfer*innen (per E-Mail) an den Prüfungsausschuss, das IPA in cc.
\item Titeländerung sind bis spätestens eine Woche vor dem Abgabetermin der schriftlichen Arbeit möglich.
\item Prüfer*innenänderung spätestens vier Wochen vor Abgabetermin der schriftlichen Arbeit möglich.
\end{itemize}
%
Anmeldung zur Prüfung des studienabschließenden Moduls (UdK) im SoSe 2025
%
\section*{Krankmeldungen / Antrag auf Verschiebung: }
%
\begin{itemize}
\item Es muss ein formloser Antrag per E-Mail mit Begründung (bei Verschiebung) bzw. ärztlichen Attest (bei Krankmeldung) an das IPA (Prüfungsausschuss und Prüfer*innen in cc) sofort nach Bekanntwerden der Verhinderung gesendet werden.
\item Ärztliche Atteste sind im Original innerhalb von drei Werktagen im IPA einzureichen (persönlich oder per Post). Eine Arbeitsunfähigkeitsbescheinigung reicht nicht aus!
\item Nach Genehmigung durch den Prüfungsausschuss wird ein neuer Termin in Absprache mit den Prüfern*innen bekannt gegeben.
\end{itemize}
%

\subsection*{Planung und Durchführung der Prüfung:}
%
\begin{itemize}
\item Die studienabschließende Prüfung besteht aus einem benoteten gestalterischen-künstlerischenwissenschaftlichen Projekt und dessen ebenfalls benoteter hochschulöffentlicher Präsentation. Das gestalterische-künstlerische-wissenschaftliche Projekt besteht aus einer praktischen Arbeit und einem schriftlichen Anteil. Beide behandeln in gegenseitiger gestalterischer-künstlerischer-wissenschaftlicher Reflexion eine von den Studierenden selbst gewählte Thematik.
\item Der Umfang des schriftlichen Anteils beträgt ca. 40 Seiten.
\item Dauer der Präsentation und mündlichen Prüfung: ca. 75 Minuten (60 Minuten Prüfung und 15 Minuten Besprechung der Benotung durch die Prüfer*innen).
\item Arbeitsabläufe müssen durch die zu Prüfenden entsprechend organisiert werden.
\end{itemize}
\$\\rightarrow\$ Sollten Präsenzprüfungen coronabedingt nicht möglich sein, werden sie online durchgeführt, der*die Erstprüfende lädt dann zu einer Webkonferenz über z.B. Webex, Zoom ein.
\begin{itemize}
\item Abgabefristen siehe Terminplan
\item Der Prüfungsplan aller stattfindenden Prüfungen ist ca. 1 Woche vor der mündlichen Prüfung als Aushang im Studio des M.A. Design \\\& Computation einzusehen. Das IPA verschickt ihn darüber hinaus als Mail, ausschließlich an die UdK-Mailadressen.
\item Prüfungsraum ist im Allgemeinen das Studio des M.A. Design \\\& Computation, andere Raumwünsche sind über die Institutsverwaltung zu erfragen und selbstständig dem PA, sowie den Prüfern*innen mitzuteilen (spätestens eine Wochen vor der mündlichen Prüfung).
\item Prüfungen finden grundsätzlich in den Räumen der UdK oder TU statt, sie dürfen nicht Bestandteil einer öffentlichen Veranstaltung sein.
\item Benötigte Geräte oder Materialien sind von den zu prüfenden Studierenden selbst bereitzustellen bzw. aus den entsprechenden Ausstattungen der Projektbereiche auszuleihen.
\item Finanzielle Unterstützung seitens der UdK oder TU kann nicht gewährt werden.
\item Nach Beendigung der Prüfung ist der Prüfungsraum schnellstmöglich zu räumen und in einem ordnungsgemäßen Zustand zu bringen.
\end{itemize}
%
Vorsitzender Prüfungsausschuss: Prof. Albert Lang, al@design-computation.berlin
Sachbearbeiter IPA: Frau Matrousian, Stud 31, stud31@intra.udk-berlin.de
Universität der Künste Berlin, Merkblatt für die Masterprüfungen im universitätsübergreifenden, forschungsorientierten, interdisziplinären M.A. Design \& Computation
für eine Anmeldung zum SoSe 2025. 
%
\end{document}